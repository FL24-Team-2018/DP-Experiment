\documentclass{osa-article}

%% Select the journal you're submitting to
%% oe, boe, ome, osac, osajournal
\journal{oe}
% Key:
% Express journals must have the correct journal selected:
% {oe} Optics Express
% {boe} Biomedical Optics Express
% {ome} Optical Material Express
% {osac} OSAC Continuum
% Other OSA journals may use:
% {osajournal} Applied Optics, Advances in Optics and Photonics, Journal of the Optical Society of America A/B, Optics Letters, Optica, Photonics Research

% Uncomment if submitting to Photonics Research.
% ONLY APPLICABLE FOR \journal{osajournal}
% \setprjcopyright

% Set the article type
\articletype{Research Article}
% Note that article type is not required for Express journals (OE, BOE, OME and OSAC)


\usepackage{graphicx}
\usepackage{relsize}
\usepackage[utf8]{inputenc}
\usepackage{hyperref}
\usepackage{bm,nicefrac,xfrac}
\usepackage{amsmath}
\newcommand{\mathup}[1]{\text{\textup{#1}}}


\begin{document}

\title{Coherence measurements with double pinholes at FLASH2}

\author{Thomas Wodzinski\authormark{1,2,*}, Mabel Ruiz-Lopez\authormark{2}, Masoud Mehrjoo\authormark{2}, Barbara Keitel\authormark{2}, Marion Kuhlmann\authormark{2}, Maciej Brachmanski\authormark{2}, Swen K\"{u}nzel\authormark{1}, Marta Fajardo\authormark{1}, and Elke Pl\"{o}njes-Palm\authormark{2}}

\address{\authormark{1}GoLP/Instituto de Plasmas e Fus\~{a}o Nuclear, Instituto Superior T\'{e}cnico, 1049-001 Lisboa, Portugal\\
\authormark{2}Deutsches Elektronen-Synchrotron DESY, Notkestrasse 85, 22607 Hamburg, Germany}

\email{\authormark{*}thomas.wodzinski@tecnico.ulisboa.pt} %% email address is required

% \homepage{http:...} %% author's URL, if desired

%%%%%%%%%%%%%%%%%%% abstract %%%%%%%%%%%%%%%%
%% [use \begin{abstract*}...\end{abstract*} if exempt from copyright]

\begin{abstract}

Since 2016 FLASH at DESY in Hamburg operates the variable-gap undulator beamline FLASH2 as a user facility. Young's double pinhole measurements were performed at photon beamline FL24 downstream of the Kirkpatrick-Baez focusing optics, which were installed in 2017. FLASH2 was characterized at wavelengths of 8, 13.5 and 18 nm and under different machine settings. The coherence length was determined from the interference pattern of several pinhole pair separationscovering the width of the beam. A blind deconvolution algorithm was implemented to determine the coherence function from the partially coherent interference pattern. Simulations of the patterns including the Kirkpatrick-Baez focusing optics were implemented with WavePropaGator (WPG), a software for X-ray wavefront propagation simulations developed at the European XFEL. We present first results of these coherence measurements and simulations.
\end{abstract}

%%%%%%%%%%%%%%%%%%%%%%%%%%  body  %%%%%%%%%%%%%%%%%%%%%%%%%%
\section{Introduction}

\subsection{Blind-deconvolution to recover the partially coherent intensity}

The measured intensity can be expressed as a convolution (see \cite[(eq.33)]{VartanyantsRobinson2001} \cite[(eq.23)]{WilliamsQuineyPeeleEtAl2007} (citing \cite{LinPatersonPeeleEtAl2003} (citing \cite{Nugent1991})), \cite[(eq.1)]{WhiteheadWilliamsQuineyEtAl2009} and \cite[(eq.5)]{ClarkHuangHarderEtAl2012}):

\[I_{\mathup{pc}} = I_\mathup{c} \ast \mathcal{F}(\gamma)\]

where $I_{\mathup{pc}}$ and $I_c$ are the partially coherent intensity and the coherent intensity, respectively, and $\mathcal{F}$ represents the Fourier transformation. $\gamma$ is termed as the normalized spatial coherence function, transverse ($r$) to the direction of the wave field propagation.

The transverse coherence length $\xi_\mathup{T}$ can be measured as the FWHM of the $\mathcal{F}(\gamma)$ at the detector plane. (reference?) 

\[\gamma \propto \exp\!\left[-\frac{r}{2\sigma}\right]\]


\[\left( I_{\mathup{pc}},\mathcal{F}(\gamma)_{\mathup{est}}\right)  \longrightarrow \left( I_\mathup{c}, \mathcal{F}(\gamma)_{\mathup{rec}} \right)  \]


%%%%%%%%%%%%%%%%%%%%%%% References %%%%%%%%%%%%%%%%%%%%%%%%%

%%%%%%%%%% If using BibTeX:
\bibliography{references}



\end{document}
